\documentclass[]{article}
\usepackage{lmodern}
\usepackage{amssymb,amsmath}
\usepackage{ifxetex,ifluatex}
\usepackage{fixltx2e} % provides \textsubscript
\ifnum 0\ifxetex 1\fi\ifluatex 1\fi=0 % if pdftex
  \usepackage[T1]{fontenc}
  \usepackage[utf8]{inputenc}
\else % if luatex or xelatex
  \ifxetex
    \usepackage{mathspec}
  \else
    \usepackage{fontspec}
  \fi
  \defaultfontfeatures{Ligatures=TeX,Scale=MatchLowercase}
\fi
% use upquote if available, for straight quotes in verbatim environments
\IfFileExists{upquote.sty}{\usepackage{upquote}}{}
% use microtype if available
\IfFileExists{microtype.sty}{%
\usepackage{microtype}
\UseMicrotypeSet[protrusion]{basicmath} % disable protrusion for tt fonts
}{}
\usepackage[margin=1in]{geometry}
\usepackage{hyperref}
\hypersetup{unicode=true,
            pdftitle={Cultural Values and Instrumental Variables - Codebook},
            pdfauthor={Ronei Leonel},
            pdfborder={0 0 0},
            breaklinks=true}
\urlstyle{same}  % don't use monospace font for urls
\usepackage{graphicx,grffile}
\makeatletter
\def\maxwidth{\ifdim\Gin@nat@width>\linewidth\linewidth\else\Gin@nat@width\fi}
\def\maxheight{\ifdim\Gin@nat@height>\textheight\textheight\else\Gin@nat@height\fi}
\makeatother
% Scale images if necessary, so that they will not overflow the page
% margins by default, and it is still possible to overwrite the defaults
% using explicit options in \includegraphics[width, height, ...]{}
\setkeys{Gin}{width=\maxwidth,height=\maxheight,keepaspectratio}
\IfFileExists{parskip.sty}{%
\usepackage{parskip}
}{% else
\setlength{\parindent}{0pt}
\setlength{\parskip}{6pt plus 2pt minus 1pt}
}
\setlength{\emergencystretch}{3em}  % prevent overfull lines
\providecommand{\tightlist}{%
  \setlength{\itemsep}{0pt}\setlength{\parskip}{0pt}}
\setcounter{secnumdepth}{0}
% Redefines (sub)paragraphs to behave more like sections
\ifx\paragraph\undefined\else
\let\oldparagraph\paragraph
\renewcommand{\paragraph}[1]{\oldparagraph{#1}\mbox{}}
\fi
\ifx\subparagraph\undefined\else
\let\oldsubparagraph\subparagraph
\renewcommand{\subparagraph}[1]{\oldsubparagraph{#1}\mbox{}}
\fi

%%% Use protect on footnotes to avoid problems with footnotes in titles
\let\rmarkdownfootnote\footnote%
\def\footnote{\protect\rmarkdownfootnote}

%%% Change title format to be more compact
\usepackage{titling}

% Create subtitle command for use in maketitle
\providecommand{\subtitle}[1]{
  \posttitle{
    \begin{center}\large#1\end{center}
    }
}

\setlength{\droptitle}{-2em}

  \title{Cultural Values and Instrumental Variables - Codebook}
    \pretitle{\vspace{\droptitle}\centering\huge}
  \posttitle{\par}
    \author{Ronei Leonel}
    \preauthor{\centering\large\emph}
  \postauthor{\par}
    \date{}
    \predate{}\postdate{}
  

\begin{document}
\maketitle

{
\setcounter{tocdepth}{4}
\tableofcontents
}
\protect\hyperlink{5HTTLPR}{5-HTTLPR}

\textbf{N\_Studies}: Number of studies that authors used to find \% S
allele\\
\textbf{N}: Total Sample of studies\\
\textbf{Percent S}: \% of S(hort) allele\\
\textbf{Perccent L}: \% of L(ong) allele

Source: Chiao, J. Y., \& Blizinsky, K. D. (2010). Culture--gene
coevolution of individualism--collectivism and the serotonin transporter
gene. Proceedings of the Royal Society B: Biological Sciences,
277(1681), 529-537.
\href{https://royalsocietypublishing.org/action/downloadSupplement?doi=10.1098\%2Frspb.2009.1650\&file=rspb20091650supp4.pdf}{Appendix
1},
\href{https://royalsocietypublishing.org/action/downloadSupplement?doi=10.1098\%2Frspb.2009.1650\&file=rspb20091650supp1.doc}{Appendix
2}

\protect\hyperlink{language1}{Pronoun Drop}

\textbf{1PS}: Number of First Personal Pronoun. 1 indicates multiples
1PS, 0 otherwise\\
\textbf{2PS}: Number of Second Personal Pronoun. 1 indicates multiples
2PS, 0 otherwise\\
\textbf{Pronoun drop}: it indicates if the language almost always
requires a 1PS pronoun in an independent clause or not. 1 indicates that
the pronoun drop is NOT allowed, zero other

Source: Kashima, E. S., \& Kashima, Y. (1998). Culture and Language.
Journal of Cross-Cultural Psychology, 29(3), 461--486.
\url{doi:10.1177/0022022198293005}

\protect\hyperlink{language2}{Linguistic Fractionalization}

\textbf{Linguistic Fractionalization}: The probability that two randomly
selected individuals from a population belong to different linguistic.

Source: Alesina, A., Devleeschauwer, A., Easterly, W., Kurlat, S., \&
Wacziarg, R. (2003). Fractionalization. Journal of Economic growth,
8(2), 155-194.

\protect\hyperlink{Religion1}{Religion - Guiso 2003}

\textbf{Catholic}: \% of Catholic\\
\textbf{Protestant}: \% of Protestant\\
\textbf{Jewish}: \% of Jewish\\
\textbf{Muslim}: \% of Muslim\\
\textbf{Hindu}: \% of Hindu\\
\textbf{Buddhist}: \% of Biddhist\\
\textbf{Other affiliations}: \% of Other affiliations\\
\textbf{No religious affiliations}: \% of No religious affiliations

Source: Guiso, L., Sapienza, P., \& Zingales, L. (2003). People's opium?
Religion and economic attitudes. Journal of monetary economics, 50(1),
225-282.

\protect\hyperlink{Religion2}{Religion - La Porta 1999}

\textbf{Catholic}: \% of Catholic in 1980\\
\textbf{Muslim}: \% of Muslim in 1980

Source: La Porta, R., Lopez-de-Silanes, F., Shleifer, A., \& Vishny, R.
(1999). The quality of government. The Journal of Law, Economics, and
Organization, 15(1), 222-279.

\protect\hyperlink{ticket}{Unpaid parking tickets}

\protect\hyperlink{fractionalization}{Ethic and religion
Fractionalization}

\textbf{Source Ethnicity Data}: The source from the ethnicity data\\
\textbf{Date Ethnicity Data}: The year for the ethnicity data\\
\textbf{Ethnic Fractionalization}: The probability that two randomly
selected individuals from a population belong to different ethnic\\
\textbf{Religion Fractionalization}: The probability that two randomly
selected individuals from a population belong to different religion\\
Note: eb = Encyclopedia Brittanica, cia = CIA, sm = Scarrit and
Mozaffar; lev = Levinson; wdm = World Directory of Minorities; census =
national census data

Source: Alesina, A., Devleeschauwer, A., Easterly, W., Kurlat, S., \&
Wacziarg, R. (2003). Fractionalization. Journal of Economic growth,
8(2), 155-194.

\protect\hyperlink{latitude}{Latitude}

\textbf{Latitude}: the absolute value of latitude in degrees divided by
90 to place it on a 0 to 1 scale. The latitude of each country was
obtained from the Global Demography Project at U.C. Santa Barbara
(\url{http://www.ciesin.org/datasets/gpw/globldem.doc.html}), discussed
by Tobler, Deichmann, Gottsegen and Maloy (1995). These location data
correspond to the center of the county or province within a country that
contains the largest number of people

Source: Hall, R. E., \& Jones, C. I. (1999). Why do some countries
produce so much more output per worker than others?. The quarterly
journal of economics, 114(1), 83-116.

\protect\hyperlink{Temperature}{Temperature}

\textbf{Temperature}: Average daytime temperature of the country's
capital city

Source: Hall, R. E., \& Jones, C. I. (1999). Why do some countries
produce so much more output per worker than others?. The quarterly
journal of economics, 114(1), 83-116.


\end{document}
